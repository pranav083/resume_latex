%%%%%%%%%%%%%%%%%%%%%%%%%%%%%%%%%%%%%%%
% Deedy - One Page Two Column Resume
% LaTeX Template
% Version 1.2 (16/9/2014)
%
% Original author:
% Debarghya Das (http://debarghyadas.com)
%
% Original repository:
% https://github.com/deedydas/Deedy-Resume
%
% IMPORTANT: THIS TEMPLATE NEEDS TO BE COMPILED WITH XeLaTeX
%
% This template uses several fonts not included with Windows/Linux by
% default. If you get compilation errors saying a font is missing, find the line
% on which the font is used and either change it to a font included with your
% operating system or comment the line out to use the default font.
% 
%%%%%%%%%%%%%%%%%%%%%%%%%%%%%%%%%%%%%%
% 
% TODO:
% 1. Integrate biber/bibtex for article citation under publications.
% 2. Figure out a smoother way for the document to flow onto the next page.
% 3. Add styling information for a "Projects/Hacks" section.
% 4. Add location/address information
% 5. Merge OpenFont and MacFonts as a single sty with options.
% 
%%%%%%%%%%%%%%%%%%%%%%%%%%%%%%%%%%%%%%
%
% CHANGELOG:
% v1.1:
% 1. Fixed several compilation bugs with \renewcommand
% 2. Got Open-source fonts (Windows/Linux support)
% 3. Added Last Updated
% 4. Move Title styling into .sty
% 5. Commented .sty file.
%
%%%%%%%%%%%%%%%%%%%%%%%%%%%%%%%%%%%%%%%
%
% Known Issues:
% 1. Overflows onto second page if any column's contents are more than the
% vertical limit
% 2. Hacky space on the first bullet point on the second column.
%
%%%%%%%%%%%%%%%%%%%%%%%%%%%%%%%%%%%%%%


\documentclass[]{font}
\usepackage{fancyhdr}
 
\pagestyle{fancy}
\fancyhf{}
 
\begin{document}

%%%%%%%%%%%%%%%%%%%%%%%%%%%%%%%%%%%%%%
%
%     LAST UPDATED DATE
%
%%%%%%%%%%%%%%%%%%%%%%%%%%%%%%%%%%%%%%
\lastupdated

%%%%%%%%%%%%%%%%%%%%%%%%%%%%%%%%%%%%%%
%
%     TITLE NAME
%
%%%%%%%%%%%%%%%%%%%%%%%%%%%%%%%%%%%%%%
\namesection{Pranav}{Kumar}{ \urlstyle{same} | Electronics, Robotics Research Enthusiast | \\
\href{mailto:kpranav083@gmail.com}{kpranav083@gmail.com} | +919614436659 | \href{https://www.linkedin.com/in/pranav083}{pranav083}
}

%%%%%%%%%%%%%%%%%%%%%%%%%%%%%%%%%%%%%%
%
%     COLUMN ONE
%
%%%%%%%%%%%%%%%%%%%%%%%%%%%%%%%%%%%%%%

\begin{minipage}[t]{0.35\textwidth} 

%%%%%%%%%%%%%%%%%%%%%%%%%%%%%%%%%%%%%%
%     EDUCATION
%%%%%%%%%%%%%%%%%%%%%%%%%%%%%%%%%%%%%%

\section{Education} 

\subsection{UIET, Panjab University}
\descript{B.E in Electronics }
\descript{and Communication}
\location{3rd Year | Chandigarh, India}
\location{CGPA: 6.96 / 10.0}
\sectionsep

\subsection{Delhi Public School}
\descript{Higher Secondary, CBSE}
\location{May 2015 | Ranchi, India}
\location{ Percentage: 90.2 / 100.0}
\sectionsep

\subsection{ABR Foundation }
\descript{Secondary, CBSE}
\location{May 2013 | Sasaram, India}
\location{ CGPA: 9.8 / 10.0}
\sectionsep

%%%%%%%%%%%%%%%%%%%%%%%%%%%%%%%%%%%%%%
%     SKILLS
%%%%%%%%%%%%%%%%%%%%%%%%%%%%%%%%%%%%%%

\section{Skills}
%\subsection{Programming}
\location{Software / Tools:}
C \textbullet{}  Embedded C \textbullet{} OpenCV \textbullet{} ROS \\
python \textbullet{} Matlab \textbullet{} RTOS   \textbullet{} Git \textbullet{} Pspice\\
Circuit Prototyping \textbullet{} Linux \textbullet{} Fritzing \\ 
\location{Hardware:}
Raspberry Pi \textbullet{} Atmel chip \textbullet{} Arduino \textbullet{} ICs \\
Circuit Design \textbullet{} I2C \textbullet{} sensor interface \\
\location{Familiar:}
Eagle \textbullet{} Gcode \textbullet{} Mcode \textbullet{} Labview \\
Tspice \textbullet{} \LaTeX\ \\
\sectionsep


%%%%%%%%%%%%%%%%%%%%%%%%%%%%%%%%%%%%%%
%     LINKS
%%%%%%%%%%%%%%%%%%%%%%%%%%%%%%%%%%%%%%

\section{Links} 
Github:// \href{https://github.com/pranav083}{\bf pranav083} \\
LinkedIn://  \href{https://www.linkedin.com/in/pranav083}{\bf pranav083} \\
Twitter://  \href{https://twitter.com/pranav083}{\bf @pranav083} \\

%%%%%%%%%%%%%%%%%%%%%%%%%%%%%%%%%%%%%%
%     COURSEWORK
%%%%%%%%%%%%%%%%%%%%%%%%%%%%%%%%%%%%%%

\section{Personal Projects}
\subsection{Harvester Bot}
\location{(Jan 2017 – Mar 2017)}
\textbullet{} A prototype machine for efficient harvesting without wastage of straw genrated in e-YIC \\
( IIT Bombay ) \\
\textbullet{} Proposal Document \underline{\textbf{\href{https://drive.google.com/file/d/0B0cF1Lq6c1cPcURJdkJjQmVFcFk3eTgtTGg5VktPVHJWWU5R/view?usp=sharing}{Link}}} \\

\vspace{\topsep} % Hacky fix for awkward extra vertical space
\subsection{2D Plotter}
\location{(June 2017 – July 2017)}
\textbullet{} With the vision of modularity among \\
prototyping hardwares, created 2D Plotter \\
using Arduino Mega, RAMP 1.4 \\
with open Source Marlin Firmware. \\
\textbullet{} Youtube \textbf{\href{https://www.youtube.com/watch?v=D679CJQNnKo&t=1s}{\underline{Video}} }
\sectionsep


%%%%%%%%%%%%%%%%%%%%%%%%%%%%%%%%%%%%%%
%
%     COLUMN TWO
%
%%%%%%%%%%%%%%%%%%%%%%%%%%%%%%%%%%%%%%

\end{minipage} 
\hfill
\begin{minipage}[t]{0.64\textwidth} 

%%%%%%%%%%%%%%%%%%%%%%%%%%%%%%%%%%%%%%
%     EXPERIENCE
%%%%%%%%%%%%%%%%%%%%%%%%%%%%%%%%%%%%%%

\section{Internship / Experience}

\runsubsection{\textit{\textbf{e-YRC (IIT Bombay)}}}
\descript{| Team Leader \& Open Source Contributer}
\location{Oct 2018 – Feb 2019 | Chandigarh, India}
\vspace{\topsep} % Hacky fix for awkward extra vertical space
\begin{tightemize}
<<<<<<< HEAD
\item Using OpenCV for object Detection and ArUco Markers. Making a lighter and less complex multifunctional line Following algorithm using FSM.
=======
\item Making a lighter and less complex multifunctional line Following algorithm
using​ ​ FSM​ that is easy to implement and build.
>>>>>>> 2ca21138c9850c8ac69777a17b0bdf702fc3d4c5
\item Youtube \textbf{\href{https://youtu.be/FhUvQlrLWxc}{\underline{Video}}} and Github \textbf{\href{https://github.com/pranav083/FSM_code}{\underline{Code}}}. 
\end{tightemize}
\sectionsep

\runsubsection{Product development for NGO}
\descript{| Student Lead }
\descript{ \textbf{Contact :} Dr. Manu Sharma - \textit{\href{mailto:manu@pu.ac.in}{\textbf{manu@pu.ac.in}}}}
\location{September 2018 – Present | UIET,PU,Chandigarh, India}
%\vspace{\topsep} % Hacky fix for awkward extra vertical space
\begin{tightemize}
\item This lab in college work as a bridge between college-industry partnership program .In this we have build many different project for NGO.
\item Design and build several product based design for NGO which show
case these in the school for encouraging secondary school.
\item Managed resourses and Team of around 20 team members. 
\item Follow up this Github Link for More Info \textbf{\href{https://github.com/pranav083/Tinkering_project}{\underline{Codes and Document}}}.	
\end{tightemize}
\sectionsep

<<<<<<< HEAD
\runsubsection{\textit{\textbf{CIC, University Of Delhi }}}
=======
\runsubsection{\textit{\textbf{CIC, Delhi University}}}
>>>>>>> 2ca21138c9850c8ac69777a17b0bdf702fc3d4c5
\descript{| Team member \& Intern}
\location{June 2018 – July 2018 | Delhi, India}
\begin{tightemize}
\item Worked on swarm robotics formation control robots using ROS(robot operating system).
\item Follow the \textbf{\href{http://crip.ml}{\underline{link}}} for more  Information.
\item Achieved work \textbf{\href{https://github.com/pranav083/ROS_work_earlier_nrf}{\underline{Code}}}.	
\end{tightemize}
\sectionsep

\runsubsection{Design and Innovation Center}
\descript{| Team Leader } 
\descript{ \textbf{Contact :} Dr. Naveen Aggarwal - \textit{\href{mailto:navagg@pu.ac.in}{\textbf{navagg@pu.ac.in}}}} 
\location{ Initiative By MHRD(Gov.of India) on collaborative research and innovation}
\location{Jan 2018 - Oct 2018 | PU,Chandigarh, India}
%\vspace{\topsep} % Hacky fix for awkward extra vertical space\begin{tightemize}
\begin{tightemize}
\item To propose a 3-D printing technology with the multipurpose
capabilities enabling us to multi tasking like 2D plotting, 3D printing etc.
\item Enable Multiple capabiltiy with only addon modules with same Base .
\end{tightemize}
\sectionsep

%%%%%%%%%%%%%%%%%%%%%%%%%%%%%%%%%%%%%%
%     Achievements
%%%%%%%%%%%%%%%%%%%%%%%%%%%%%%%%%%%%%%

\section{Achievements} 
\begin{tabular}{rll}
2018	     & Pocket Beagle at Mouser electronics event at IIT Roorkee \\
2019	     & Poster presentation at Chandigarh Science Congress(CHESCON)\\
\end{tabular}
\sectionsep

%%%%%%%%%%%%%%%%%%%%%%%%%%%%%%%%%%%%%%
%     Organization And Events
%%%%%%%%%%%%%%%%%%%%%%%%%%%%%%%%%%%%%%
\section{Organization And Events}
\runsubsection{Embedded Group of UIET}
\descript{| Community Head}
\location{June 2017 – present | \textit{\textbf{\href{https://t.me/eguiet}{\underline{Telegram Group Link}}} }}
\location{ Initiative to develop a open embedded environment for all with around 340+ members.}
\sectionsep

\vspace{\topsep} % Hacky fix for awkward extra vertical space
\runsubsection{Workshops and Seminar}
\descript{| Student Speaker}
\location{June 2017 – present |  \textit{\textbf{\href{https://drive.google.com/drive/folders/1DxT2_Fzj40LQEiOTwckY73_On0V8weF_?usp=sharing}{\underline{Drive Link}}}}}
\location{\textit{Organized different WORKSHOP and SEMINAR on different Topic related to electronics.}}
\sectionsep

\vspace{\topsep} % Hacky fix for awkward extra vertical space
\runsubsection{Electronics Hackathon}
\descript{| Organizer}
\location{June 2017 , August 2018 | \textit{\textbf{\href{https://photos.app.goo.gl/6EWs1pcrRarGybta9}{\underline{Photo Gallery Link}}}}}
\location{Organized electronics hackathon on North India level.}
\sectionsep

\end{minipage} 
\end{document}  \documentclass[]{article}